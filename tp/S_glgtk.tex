\section{Intégration d'un espace de rendu OpenGL dans une fenêtre GTK}
Lorsque vous souhaitez utilliser OpenGL dans un projet usillisant 
LablGTK pour le fenêtrage, il n'existe qu'un chemin sûr à prentdre: le module GlGtk!
\subsection{Ajout de l'espace de dessin dans notre fenêtre}
Le widget qui servira d'espace de dessin est le widget {\bf area}
du module {\bf GlGtk}. Il possède les même paramêtres que tous
les widgets de LablGtk, avec en plus des flags d'initialisation de
l'environnement d'OpenGL: le format des couleurs, le double buffering,
etc...\\
Par exemple, vous pouvez inisialliser votre widget de la manière suivante:

\begin{lstlisting}
(* Some code before *)
let glWindow = GlGtk.area
		~width:500
		~height:300
		~packing:fenetre#add
		~show:true
		[
		'RGBA;
		'DOUBLEBUFFER;
		'BUFFER_SIZE(8);
		'DEPTH_SIZE(16)
		]
		()
in
(* Some code after  *)
\end{lstlisting}
\subsection{Dessiner un cube}
Maintenent que vous avez la GlArea insérée dans votre fenêtre, il
ne reste plus qu'à dessiner à l'intérieur.\\
En realité, il n'y a pas grand chose à faire d'autre que des
appels aux fonctions de LablGl. Toutefois, vous devez précéder 
toutes vos séries d'appel aux fonctions de LablGl par l'appel
à la procédure {\bf make\_current} disponible à travert votre
objet de type GlGtk.area.\\

De plus, il vous faudra remplacer l'appel au swap buffer de LablGl
par la méthode {\bf swap\_buffers} de votre widget GlGtk.area.\\
Ainsi, vous optiendrez un code ressemblant à ceci:\\

\begin{lstlisting}
(* GlGtk.area object has been created with the name: glWin *)
let displayCube () =
	glWin#make_current ();
	(* Appel aux fonctions de LablGl *)
	glWin#swap_buffers ()
\end{lstlisting}

Il ne vous reste plus qu'à dessiner un cube! (ou toute autre forme
de votre choix...).
\subsection{La souris dans votre espace de dessin}
	Il reste tout de même un problême! Comment faire pour
	usilliser la souris dans votre widget area?\\
	En fait, c'est impossible: le widget de dessin ne supporte
	pas les entrée souris/clavier. C'est pour cela qu'il faut
	trouver un moyen de contourner cela: utilliser une
	{\bf event\_box} du module GBin et y insérer votre widget
	{\bf GlGtk.area}.
%\subsection{Modifier le Makefile}
